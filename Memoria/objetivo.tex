\chapter{Objetivos}
A continuaci�n se explicar�n de forma general y detallada los objetivos propuestos para este \ac{TFM}.

\section {Objetivo general}
El objetivo general del proyecto consiste en el dise�o e implementaci�n de una soluci�n que permita la generaci�n de documentos PDF a partir de la informaci�n obtenida de los productos de una plataforma de comercio electr�nico. Un administrador de un portal de comercio electr�nico debe ser capaz de gestionar la disposici�n de los elementos que componen un producto (fotograf�a, descripci�n, t�tulo, precio, referencia, etc.) del PDF a generar de una forma �gil y sencilla. Todo este proceso ser� transparente para un usuario final, que �nicamente con presionar un bot�n obtendr� la informaci�n de los productos en un documento PDF.

\section {Objetivos espec�ficos}
Para alcanzar el objetivo general del proyecto es necesario cubrir los siguientes objetivos espec�ficos:

\begin{itemize}
	\item Elegir una plataforma base entre \acs{LAMP} y \acs{WAMP} para el desarrollo del \ac{TFM}
	\item Instalar y aprender a utilizar el \ac{CMS} Drupal cogiendo soltura con sus elementos: vistas, tipos de contenido, m�dulos, etc.
	\item Coger soltura con el lenguaje de programaci�n PHP.
	\item Utilizar Ubercart como extensi�n de comercio electr�nico para la realizaci�n del \ac{TFM}.
	\item Permitir que el administrador de la plataforma de comercio electr�nico pueda gestionar distintas plantillas PDF.
	\item Permitir la edici�n �gil del contenido de una plantilla PDF en el lado del cliente.
	\item Elegir una tecnolog�a para el lado del cliente que permita crear un interfaz de usuario �gil.
	\item Permitir que un usuario an�nimo genere documentos PDF a partir de productos y cat�logos de productos.
	\item Crear una soluci�n exportable a otras instalaciones de Drupal mediante la creaci�n de un m�dulo
\end{itemize}