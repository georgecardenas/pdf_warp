\chapter{Resumen}
%\thispagestyle {empty}

    En la actualidad el mundo del comercio electr�nico est� viviendo una edad de oro. Empresas como Amazon obtienen al a�o m�s de 100.000 millones en ventas on-line con alrededor de 300 millones de clientes en todo el mundo.\\
    
    Uno de los problemas de las plataformas de e-commerce actuales, radica en que un cliente necesita en ocasiones guardar informaci�n sobre un producto para su posterior consulta, y, por lo general, la presentaci�n de �stos productos una vez impresos o guardados en un documento, como por ejemplo PDF, no es la m�s adecuada. Un administrador de la plataforma, no suele contar con medios ni formaci�n para poder modificar la presentaci�n de sus productos a la hora de generar un documento, teniendo que recurrir a programadores especializados que realicen la maquetaci�n de los mismos.\\
    
    Como respuesta a este problema, se propone realizar un m�dulo para una plataforma de e-commerce conocida con el que un administrador pueda crear plantillas de documentos PDF de forma r�pida y sencilla de modo que un usuario pueda guardar la informaci�n de los productos en PDF de acuerdo a las mismas.
    
\vspace{2cm}
{\textcolor{blue}{\large\bf PALABRAS CLAVE}}\\
E-commerce, PDF, M�dulo, Plantilla