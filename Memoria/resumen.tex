\chapter{Resumen}
%\thispagestyle {empty}

    En la actualidad el mundo del comercio electr�nico est� viviendo una edad de oro. Empresas como Amazon obtienen al a�o m�s de 100.000 millones en ventas \textit{online} con alrededor de 300 millones de clientes en todo el mundo.\cite{Dav16}\\
    
    Uno de los problemas de las plataformas de comercio electr�nico actuales, radica en que un cliente necesita, en ocasiones, guardar informaci�n sobre un producto para su posterior consulta, y, por lo general, la presentaci�n que se obtiene al imprimir los productos mediante el navegador, por ejemplo en formato PDF, no es la m�s adecuada. En particular es habitual que el producto aparezca mezclado con elementos HTML de la p�gina, que nada tienen que ver con el producto en s�. Un administrador de una plataforma de comercio electr�nico, no suele contar con medios ni formaci�n para poder modificar la presentaci�n de sus productos a la hora de generar un documento, teniendo que recurrir a programadores especializados que realicen la maquetaci�n de los mismos.\\
    Otro problema relacionado con el anterior es que, pese a la gran inserci�n actual del comercio electr�nico, muchos negocios online disponen tambi�n de tiendas f�sicas donde los clientes reclaman cat�logos en papel. En este sentido, ser�a de utilidad poder generar un cat�logo en PDF que incluya no solo un producto, sino el cat�logo completo.\\
    
    Como respuesta a este problema, se propone realizar una extensi�n para una plataforma de comercio electr�nico conocida con la que un administrador pueda crear plantillas de documentos PDF de forma r�pida y sencilla de modo que un usuario pueda guardar la informaci�n de los productos en PDF de acuerdo a las mismas.\\ 
    En concreto, para la realizaci�n de este Trabajo Fin de M�ster, se ha escogido la plataforma Drupal junto con su extensi�n de comercio electr�nico Ubercart.
    
\vspace{1cm}
{\textcolor{blue}{\large\bf PALABRAS CLAVE}}\\
Comercio electr�nico, PDF, Extensi�n, Drupal, Cat�logo, Producto