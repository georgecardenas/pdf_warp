\chapter{Conclusiones y trabajos futuros}

Durante la realizaci�n de este \ac{TFM} se han integrado con �xito diversas tecnolog�as que trabajan en conjuto para lograr un objetivo. En este caso concreto, y a modo de recordatorio, el objetivo era la generaci�n de documentos PDF a partir de la informaci�n obtenida de los productos de una plataforma de comercio electr�nico y las plantillas que haya creado un usuario administrador. Bajo el punto de vista de los objetivos, se puede decir que el proyecto ha resultado un �xito ya que se cumplen tanto el objetivo general, como los objetivos espec�ficos y se ha conseguido realizar una extensi�n para Drupal completamente funcional y exportable, aun partiendo de un nivel de conocimientos bajo en PHP y Drupal.\\

Con respecto a la metodolog�a de trabajo utilizada, no se han encontrado mayores dificultades a la hora de aplicar el proceso Scrum al �mbito de un \ac{TFM}, en realidad no se requiere de la implementaci�n de cambios mayores de la forma de trabajar en una colaboraci�n tutor/alumno y se siente que es la forma de trabajar adecuada para un proyecto de estas caracter�sticas. El \textit{product backlog} es una gran herramienta ya que, incluso con el limitado tiempo con el que se ha contado en este caso, permite organizar el trabajo de manera r�pida y sencilla, eligiendo que tareas se desempe�ar�n para cada entrega.\\

Como se ha comentado, una dificultad encontrada ha sido el tiempo disponible. Habiendo dispuesto de m�s tiempo, se podr�a haber realizado una aplicaci�n m�s s�lida, con un dise�o m�s atractivo y con m�s funcionalidades.\\

Este proyecto podr�a ser continuado y extendido por otro alumno, tanto para un trabajo de una asignatura como para un proyecto propio, de modo que se proponen las siguientes mejoras:
\begin{itemize}
	\item Elegir de forma sencilla que elementos aparecen en una plantilla, ocultando los que no queremos mostrar.
	\item Generaci�n de los elementos PDF gen�ricos. Actualmente funciona para los elementos b�sicos de un producto.
	\item Adici�n de atributos de Ubercart a los documentos PDF. 
	\item Permitir modificar la plantilla de producto dentro de un cat�logo.
	\item Redimensi�n de productos en cat�logos.
	\item Men�s de elementos de plantilla contextuales en vez de fijos.
	\item Redimensi�n din�mica arrastrando desde los bordes del elemento.
	\item Permitir la selecci�n de tipo de fuente para el PDF.
	\item Permitir a�adir la portada y contraportada desde otro documento PDF.
	\item Permitir a�adir cabecera y pie a un PDF.
	\item Visor PDF integrado en Drupal.
	\item Dise�o mejorado y atractivo.
\end{itemize}